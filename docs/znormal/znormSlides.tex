\documentclass{beamer}
\usepackage{amsmath}
\usepackage{amssymb}
\title{$\mathbb{Z}$ Normal Distribution}
\author{Zach Sullivan}
\date{2017}
\usecolortheme{seagull}
\setbeamertemplate{itemize items}[circle]

\begin{document}
\frame{\titlepage}

\begin{frame}
\frametitle{Motivation}
Western music is discretized so that it is easier to notate.
\vspace{1em}

Discrete, yet infinitely negative and positive features:
\begin{itemize}
\item octaves
\item dynamics
\item tempo
\item duration
\end{itemize}
\end{frame}

\begin{frame}
\frametitle{Normal Distribution}
\[ \mu \in \mathbb{R},\ \sigma \in \mathbb{R}^+
\]
\vspace{1em}
\[ p(x \mid \mu, \sigma^2) =
  \frac{1}{\sqrt{2\pi\sigma^2}} e^{-\frac{(x - \mu)^2}{2\sigma^2}}
\]
\vspace{1em}
\[ 1 = \int_\mathbb{R} p(x \mid \mu, \sigma^2)
\]
\end{frame}

\begin{frame}
\frametitle{$\mathbb{Z}$ Normal Distribution}
\[ \mu \in \mathbb{Z},\ \sigma \in \mathbb{R}^+
\]
\vspace{1em}
\[ p(x \mid \mu, \sigma^2) =
  \frac{1}{\sqrt{2\pi\sigma^2}} e^{-\frac{(x - \mu)^2}{2\sigma^2}}
\]
\vspace{1em}
\[ 1 = \sum_\mathbb{Z} p(x \mid \mu, \sigma^2)
\]
\end{frame}

\begin{frame}
\frametitle{$\mathbb{Z}$ Normal Distribution}
Sample from $\mathcal{N}_{\mathbb{Z}}$ distribution with a mean
$\mu \in \mathbb{Z}$ and a standard deviation of $\sigma^2 \in \mathbb{R}^+$
\begin{align*}
&\ \,x \sim \mathcal{N}(\mu,\sigma^2)\\
&\lfloor x \rfloor
\end{align*}
\end{frame}

\begin{frame}
\frametitle{$\mathbb{Z}$ Normal Distribution}
{\it Theorem 1.} This sampling method for $\mathcal{N}_{\mathbb{Z}}$ distribution
denotes a probability distribution over $\mathbb{Z}$.
\end{frame}

\begin{frame}
\frametitle{{\it Theorem}}
{\it Proof.}
\end{frame}

\begin{frame}
\frametitle{Durations}
Most common durations are half, quarter, and eight, but the scale is infinite
\vspace{1em}

Durations can be represented as a pair in $\mathbb{Z} \times \mathbb{N}$ where
the first element is the base duration and the second element is the number of
dots.

\begin{align*}
d &\sim \mathcal{N}_{\mathbb{Z}}(\mu,\sigma^2)\\
x &\sim B(n,p)\\
(d&,x)
\end{align*}
\end{frame}

\begin{frame}
\frametitle{Octaves}
\end{frame}

\begin{frame}
\frametitle{Tempo}
\end{frame}

\begin{frame}
\frametitle{Dynamics}
\end{frame}
\end{document}